\documentclass{article}
\usepackage{geometry}
\usepackage[hidelinks]{hyperref}
\usepackage{enumitem}

\geometry{a4paper, margin=1in}

\begin{document}
	\title{Infection Spread Simulation Project - Project Requirements}
	\author{}
	\date{\today}
	\maketitle
	\newpage
	\tableofcontents
	\newpage
	
	\section{Objective}
	The primary goal of this project is to create a simulation that accurately portrays the spread of infection. Within the simulation, individuals will be generated, move throughout the environment, and face the possibility of infection within a specified radius of other infected individuals. Each person in the simulation will be assigned different statuses, such as healthy, infected, or isolated, that only happen with a certain percentage. Additionally, the project necessitates the development of a user interface (UI) to visually display the simulation in real-time.
	
	\section{Project Background}
	The project revolves around simulating the spread of an infection. It involves incorporating various statuses for individuals and providing a visual representation through a user interface.
	
	\section{Requirements}
	
	\subsection{Functionalities}
	The project requires the implementation of several key functionalities. Firstly, it must simulate the generation and movement of people within the environment. The simulation must provide the opportunity to be stopped and continued at any time. Additionally, the simulation must incorporate the spread of infection within a specified radius. Furthermore, individuals must be assigned different statuses, such as healthy, infected, or isolated. The project demands the development of a user interface for the real-time display of the simulation. Lastly the project requires documentation of the mathematical background.
	
	\subsection{Performance Requirements}
	The project imposes specific performance requirements. It must ensure efficient processing of the simulation and provide either real-time or near-real-time representation of the spread of infection.
	
	\subsection{User-Friendly Interface}
	The user interface should be designed with a focus on user experience. This involves developing an intuitive UI for simulation display and providing clear documentation to guide users through the interface.
	
	\subsection{Compatibility}
	The project must ensure compatibility with different platforms for the user interface, catering to a diverse range of users and systems.
	
	\subsection{Maintenance}
	The design of the project should prioritize ease of maintenance for future updates, ensuring that the system can be readily adapted to incorporate new features or address issues.
	
	\subsection{Legal Requirements}
	To comply with legal standards, the project must adhere to licensing standards for the software used in its development.
	
	\section{Use Cases}
	The end-user envisions a simulation system that effectively represents the spread of infection, considering various individual statuses.
	
	\section{Interfaces}
	The project requires the development of a user interface specifically designed for the display of the simulation. The simulants need to be displayed in different colors according to their statuses. The UI should include the option to enter custom parameters and controls to start and stop the simulation. Furthermore, the UI should show the statuses with the amount of simulants that currently have that status.
	
	\section{Data Requirements}
	The project necessitates the collection and utilization of simulation data, including individual statuses and movements.
	
	\section{Quality Requirements}
	The project's success hinges on the reliable representation of infection spread, as well as acceptable simulation speed. The documentation needs to include a detailed description of the algorithms for status changes, movements, etc.
	
	\section{Timeline and Milestones}
	The project must be completed by March 2024, with specific milestones to track progress.
	
	\section{Risks and Assumptions}
	Potential risks include challenges in simulation accuracy, while assumptions are based on the expectation of stable system performance. There is also a risk of the simulation not being displayed in real time.
	
	\section{Approval}
	This document receives initial approval from Mrs Rose, and any subsequent changes require consultation.
	
	\section{Attachments}
	No attachments are available at the moment.
	
\end{document}
